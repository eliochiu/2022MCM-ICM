\section{Strength and Possible Improvement}
\subsection{Strength}
\begin{itemize}

\item \textbf{Comprehensive consideration:} Our model considers comprehensive aspects of the developing process and management aspects. Not only simulates the natural carbon sequestration, out model also takes economical value and cultural value into the consideration to fully maximize the forests' value in various aspects, and using money to give the final quantitative indicators.

\item \textbf{Well-Simulated, complete parameter system:} Supported by multidimensional parameters, our \textbf{SHG} model and \textbf{EEC} model are capable of simulating complicated cases with multi parameters.

\item \textbf{Self-adjustment:} Harvest will reduce the living stock in the forest, but will enhance the carbon sequestration rate. With the negative feedback considered in the model, our model can make self-adjustment by comparing the history data to make decision of adjusting the coefficient in the system. This advantage can give important instruction to the forest managers.

\end{itemize}
\subsection{Possible Improvements}
\begin{itemize}

\item \textbf{Short-period prediction:} In our model and collected data, the minimum time units are all measured in year. Besides, it's difficult to obtain the monthly data from the authority. Thus, the short-period prediction is still need to be developed. 

\item \textbf{Average harvest:} In the carbon sequestration model, we assume that the timber product companies will make average harvest of the selected area. However, when the area given by the schedule is too large to fully cover, the worker may choose to harvest in an smaller area to save time and gasoline, which may led to the deviation of the predicted values from the actual data.

\item \textbf{Dataset dependency}: This shortcoming comes from the decision tree model. Due to the complexity of our indicators and the limited amount of data, only 1,550 forest samples are finally trained, and the threshold for decision-making may be biased.
\end{itemize}

%后期使用pdf编辑器处理