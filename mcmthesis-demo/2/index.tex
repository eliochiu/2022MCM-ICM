\section{Assumptions and Justifications}
\begin{comment}
To simplify the problem, the following assumptions are made and justified.
\end{comment}

\begin{itemize}
  \item \textbf{Assumption 1:} The statistical data obtained are reliable and accurate.
	\begin{itemize}
  	 \item[$\hookrightarrow$] \textbf{Justification:} Most data are collected from authoritative websites and papers, under which our model is operational and practical.
 	\end{itemize}
  \item \textbf{Assumption 2:} The forest ecosystem is the smallest unit we consider.
        \begin{itemize} 
          \item[$\hookrightarrow$] \textbf{Justification:} Forest ecosystems are incredibly complicated. To preserve the macroscopic nature of the model, all data are collected on the principle that we view the forest as a whole.
        \end{itemize}
  \item \textbf{Assumption 3:} No massive unlawful logging activity will be carried out during the managing process, and the timber harvesting processes are all within control.
        \begin{itemize}
          \item[$\hookrightarrow$] \textbf{Justification:} During the regulating process, the executors will strictly stick to the plan, and the security of the forest resources can be guaranteed, which means there will not be any massive illegal timber harvesting process.
        \end{itemize}
  \item \textbf{Assumption 4:} Extreme events are ignored, which means fixed parameters are stable, such as the area of the forest, the structure of forest product, etc.
        \begin{itemize}
          \item[$\hookrightarrow$] \textbf{Justification:} Since the forest is stationary, we assume the biosphere is stable and developing. The force majeure will not be considered. 
        \end{itemize}
\end{itemize}