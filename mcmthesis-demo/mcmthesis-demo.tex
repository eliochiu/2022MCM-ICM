%%
\documentclass{mcmthesis}
\mcmsetup{
        CTeX = false,   % 使用 CTeX 套装时,设置为 true
        tcn = 2213962, 
        problem = E,
        sheet = true, 
        titleinsheet = true,
        keywordsinsheet = true,
        titlepage = false,
        abstract = false
        }
        
\usepackage{mathpazo} 
\usepackage{lipsum}
\usepackage{algorithm}
\usepackage{bm}
\usepackage{algorithmicx}
\usepackage{algorithmic}
\usepackage{cite}
\usepackage{float}
\usepackage{subfigure}
\usepackage{multicol}
\usepackage{pdfpages}
\usepackage{indentfirst}
\usepackage{tikz}
\usepackage{pgfplots}
\usepackage{hyperref}
\usepackage{lastpage}
\usepackage{enumitem}
\usepackage{siunitx}
\usepackage{amssymb}
\usepackage{pgfplots}
\usepackage{multirow}
\usepackage{microtype}
\usepackage{textcomp}
\usepackage{gensymb}
\usepackage{vwcol}  
\usepackage{tabulary}
\usepackage{supertabular}
\usepackage{pgf-pie}
\usepackage{tabu}
\usepackage{verbatim}
\usepackage{graphicx}
\usepackage{fitbox}
\usepackage{wrapfig}


\definecolor{color1}{RGB}{251, 238, 172}
\definecolor{color2}{RGB}{244, 209, 96}
\definecolor{color3}{RGB}{138, 196, 208}
\definecolor{color4}{RGB}{40, 82, 122}
\definecolor{color5}{RGB}{157, 83, 83}
\definecolor{color6}{RGB}{99, 38, 38}
\definecolor{darkblue}{rgb}{0.0, 0.0, 0.55}
\definecolor{darkgray}{rgb}{0.66, 0.66, 0.66}

\setlength{\headheight}{13.6pt}

\title{\textbf{Dynamic Forest \& Brighter Future}}
\begin{document}
\begin{abstract}

%1. Introduce the background of the problem and be aware of you assignment.
%%简要引入背景,清晰交代任务。
In recent years, with global warming and other climate changes, the concept of "carbon" has attracted widespread public attention. Low-carbon-emission products such as clear energy has prevailed a lot. However, in addition to reducing carbon emissions, carbon sequestration is also a powerful way to mitigate carbon dioxide. With an attempt to effectively increase the amount of carbon sequestration and provide possible strategies for forest managers, our work is proceeded as follows.

%2. Point out the mathematical model you built, describe the purpose of your model and how it will be implemented.
%%指出你针对哪一任务,用了哪些模型、模型的目的以及实现的方法。
First of all, we establish a dynamic \textbf{SHG model} to simulate Sequestering-Harvesting-Growing carbon cycle of a forest ecosystem. A forest is divided into three parts: \textbf{product}, \textbf{mature} and \textbf{immature} area. They appear to be independent but influence each other. The model takes fixed parameters of the forest (area, tree species, region, etc.) as input and outputs the total amount of carbon sequestered over a period of time. By the Monte Carlo method, we obtain a haversting rate of \textbf{3\% to 5\% }that maximizes carbon sequestration.

%3. Give specific answers in the order of the questions.
%%每一步建模任务中,要依照问题顺序给出具体解答数字。
Secondly, we devise a comprehensive and practical indicator system, on which the \textbf{ECC model} is based. ECC model provides a value quantitative method in which we divide the values of forests into three categories: \textbf{ecological}, \textbf{economic} and \textbf{cultural} values. Among these, ecological value depends on the result of SHG model.

%4. Sensitivity and robustness analysis, possible model improvement.(if possible)
%%灵敏度、鲁棒性分析以及模型改进的可能。
Thirdly, according to the result of ECC model, we develop Target Oriented Forest Management Strategy by adopting \textbf{decision tree algorithm}. With 2,214 forest samples, we calculate the percentage of different values and put them as features into the algorithm, which automatically generates a decision tree with threshold. The final decision threshold value is \textbf{10.8\%} of economic value percentage and \textbf{69.5\%} of ecological value percentage. They can be viewed as a transition point in forest management. Forest managers can assess the current state of forest by calculating the combination of different values and compare it with the threshold to determine the future development directions. 

Finally, we apply our models in Heolongjiang Province of China and obtain the total 100-year carbon sequestration of $\bm{7.599\times 10^9}$ tonnes. During 2010 to 2018, the total value of it reaches to \textbf{92.2249} billion dollars in 2018 where ecological, economic and cultural value account for \textbf{81.64\%}, \textbf{12.68\%} and \textbf{5.67\%} respectively. According to our strategy, forests of Heilongjiang are in intermediate stage and they should adopt a culture-oriented strategy to develop tourism and enhance the input of scientific research. The transition point of management emerged in \textbf{2014-2015}. 

\begin{keywords}
Carbon Sequestration; SHG; Value Quantification; Decision-making
\end{keywords}
\end{abstract}
\maketitle

%% Generate the Table of Contents, if it's needed.
\tableofcontents
\newpage

\section{Case Study}

Since we are required to provide forest managers with reasonable forest management plans to understand the best use of forests, we combine \textbf{SHG Simulation Model} and\textbf{ EEC Value Model }with specific examples to comprehensively consider and calculate the ecological, economic and social values of forests. Then, we use the \textbf{Target Oriented Forest Management Strategy} to make sure harvest is included in the forest management plan and determine the future management plan most suitable for the current forest development.

\subsection{Forests of Heilongjiang Province,China}

Heilongjiang is located in Northeast China with vast forests which can bring huge benefits for local people. 

\begin{figure}[H]
\centering
\includegraphics[scale = 0.08]{mcmthesis-demo/figures/heilongjiang.png}
\caption{Forest distribution map of Heilongjiang Province} 
\end{figure}

We log on the official websites of China Meteorological Administration and Forestry Administration to obtain the data required in EEC Value Model from 2010 to 2018, and quantify the forest value of Heilongjiang according to its formula. As shown in the table below.

\begin{table}[h]
	\centering
%   \vspace*{-14cm}              % 调整图像上间距
	\caption{Quantitative forest value in Heilongjiang from 2014 to 2016(billion dollars)}
	\tabulinesep=1mm
	\begin{tabu}to \linewidth{X[0.9,c,m]X[1.4,c,m]X[1.4,c,m]X[1.4,c,m]}
		\tabucline[0.08em]-
		year           & ecologic value    & economic value  &cultural value         \\\tabucline[0.08em]-
		2010      & 55.9578 & 6.9057 & 3.47850 \\
        2011      & 57.2133 & 7.3199 & 3.7113 \\
        2012      & 59.7257 & 7.6231 & 3.9765 \\
        2013      & 61.7045 & 7.8787 & 4.3097 \\
        2014	   & 62.8302      & 8.0561    &4.54694  \\
		2015	   & 67.7652   & 9.6319    &4.5624  \\
		2016	   & 73.0542    &11.1379  &4.5402   \\
        2017       & 74.2948  & 11.5136  &4.9423 \\
        2018      &  75.3012  & 11.6933  &5.2304 \\
        
		\tabucline[0.08em]-
	\end{tabu}
%   \vspace*{-5mm}              % 调整图像下间距
    \label{table5}
\end{table}

\subsubsection{Carbon sequestration stock in the next 100 years}

According to the SHG Simulation Model, we get the maximum annual carbon dioxide storage under the optimal proportion of forest harvest. After accumulation, the total amount of carbon dioxide that can be stored in Heilongjiang forest within 100 years can be obtained. The annual $TCS$ and $\alpha$ is plotted in Fig. [\ref{Forecast line chart}].
\begin{figure}[H]
\centering
\includegraphics[height = 9cm, width = 14cm]{mcmthesis-demo/figures/Stock.png}
\caption{Forecast line chart of carbon sequestration and proportion of tree harvest}
\label{Forecast line chart}
\end{figure}
According to our model, when the initial $\alpha =0.01767 $ and $\beta = 2.286 $ ,the total amount of the carbon sequestration over 100 years will reach its maximum of $7.599\times 10^9$ tonnes, 15.6 \% more than carbon sequestration taken by untouched forest. So the harvest should be considered in the management plan to maximize the amount of annual carbon sequestration.

\subsubsection{Forest Management Plan in Heilongjiang Forest}

\begin{table}[b]
	\centering
%   \vspace*{-14cm}              % 调整图像上间距
	\caption{the proportion of different forest values in Heilongjiang from 2010 to 2018}
	\tabulinesep=1mm
	\begin{tabu}to \linewidth{X[0.9,c,m]X[1.4,c,m]X[1.4,c,m]X[1.4,c,m]}
		\tabucline[0.08em]-
		year           & ecologic value    & economic value  &cultural value         \\\tabucline[0.08em]-
		2010      &84.34\% & 10.41\% & 5.24\% \\
        2011      & 83.83\% & 10.72\% & 5.43\% \\
        2012      & 83.73\% & 10.68\% & 5.58\% \\
        2013      & 83.50\% & 10.66\% & 5.83\% \\
        2014	   & 83.76\% & 10.67\%   &6.03\% \\
		2015	   & 82.68\%   & 11.75\%   &5.57\% \\
		2016	   & 82.33\%    &12.55\%  &5.11\%   \\
        2017       & 81.87\%  & 12.68\%  &5.45\% \\
        2018      &  81.64\%  & 12.68\%  &5.67\% \\
		\tabucline[0.08em]-
	\end{tabu}
%   \vspace*{-5mm}              % 调整图像下间距
    \label{table6}
\end{table}

Based on the data in Tab. [\ref{table5}], we can calculate the proportion of different forest values in Heilongjiang from 2010 to 2018. We apply the decision tree algorithm to make the analysis about Tab. [\ref{table6}] below, and use the Target Oriented Forest Management Strategy to give proper suggestion on the future direction of forest development.All the conclusions and advice are stated below.



	\begin{itemize}
\item \textbf{Conclusion 1:} Heilongjiang forest has been in the intermediate stage since 2010 according to the decision tree algorithm. The economic value ratio was stable at around 10.5\% and the ecological value ratio was above 80\% for 4 years which indicates that Heilongjiang forest was protected well at that time without deforestation or overuse.
\item \textbf{Conclusion 2:} The transition point during 2010 to 2018 in Heilongjiang forest is 2015 when the economic value ratio was more than 10.8\% and the ecological value ratio was still above 80\%, a symbol of an advanced stage in decision tree algorithm. 
\item \textbf{Suggestion:} Heilongjiang forest can be seen as a stable and advanced forest since 2014 with few possibility that the proportion may change accidently, so the development strategy can be directly into culture-oriented one. Government should explore the underlying value of the forest in cultural layer such education, tourism and scientific research.
	\end{itemize}
    

%第一章:简介
%主要包括:问题背景、问题复述、解决方案

\section{Case Study}

Since we are required to provide forest managers with reasonable forest management plans to understand the best use of forests, we combine \textbf{SHG Simulation Model} and\textbf{ EEC Value Model }with specific examples to comprehensively consider and calculate the ecological, economic and social values of forests. Then, we use the \textbf{Target Oriented Forest Management Strategy} to make sure harvest is included in the forest management plan and determine the future management plan most suitable for the current forest development.

\subsection{Forests of Heilongjiang Province,China}

Heilongjiang is located in Northeast China with vast forests which can bring huge benefits for local people. 

\begin{figure}[H]
\centering
\includegraphics[scale = 0.08]{mcmthesis-demo/figures/heilongjiang.png}
\caption{Forest distribution map of Heilongjiang Province} 
\end{figure}

We log on the official websites of China Meteorological Administration and Forestry Administration to obtain the data required in EEC Value Model from 2010 to 2018, and quantify the forest value of Heilongjiang according to its formula. As shown in the table below.

\begin{table}[h]
	\centering
%   \vspace*{-14cm}              % 调整图像上间距
	\caption{Quantitative forest value in Heilongjiang from 2014 to 2016(billion dollars)}
	\tabulinesep=1mm
	\begin{tabu}to \linewidth{X[0.9,c,m]X[1.4,c,m]X[1.4,c,m]X[1.4,c,m]}
		\tabucline[0.08em]-
		year           & ecologic value    & economic value  &cultural value         \\\tabucline[0.08em]-
		2010      & 55.9578 & 6.9057 & 3.47850 \\
        2011      & 57.2133 & 7.3199 & 3.7113 \\
        2012      & 59.7257 & 7.6231 & 3.9765 \\
        2013      & 61.7045 & 7.8787 & 4.3097 \\
        2014	   & 62.8302      & 8.0561    &4.54694  \\
		2015	   & 67.7652   & 9.6319    &4.5624  \\
		2016	   & 73.0542    &11.1379  &4.5402   \\
        2017       & 74.2948  & 11.5136  &4.9423 \\
        2018      &  75.3012  & 11.6933  &5.2304 \\
        
		\tabucline[0.08em]-
	\end{tabu}
%   \vspace*{-5mm}              % 调整图像下间距
    \label{table5}
\end{table}

\subsubsection{Carbon sequestration stock in the next 100 years}

According to the SHG Simulation Model, we get the maximum annual carbon dioxide storage under the optimal proportion of forest harvest. After accumulation, the total amount of carbon dioxide that can be stored in Heilongjiang forest within 100 years can be obtained. The annual $TCS$ and $\alpha$ is plotted in Fig. [\ref{Forecast line chart}].
\begin{figure}[H]
\centering
\includegraphics[height = 9cm, width = 14cm]{mcmthesis-demo/figures/Stock.png}
\caption{Forecast line chart of carbon sequestration and proportion of tree harvest}
\label{Forecast line chart}
\end{figure}
According to our model, when the initial $\alpha =0.01767 $ and $\beta = 2.286 $ ,the total amount of the carbon sequestration over 100 years will reach its maximum of $7.599\times 10^9$ tonnes, 15.6 \% more than carbon sequestration taken by untouched forest. So the harvest should be considered in the management plan to maximize the amount of annual carbon sequestration.

\subsubsection{Forest Management Plan in Heilongjiang Forest}

\begin{table}[b]
	\centering
%   \vspace*{-14cm}              % 调整图像上间距
	\caption{the proportion of different forest values in Heilongjiang from 2010 to 2018}
	\tabulinesep=1mm
	\begin{tabu}to \linewidth{X[0.9,c,m]X[1.4,c,m]X[1.4,c,m]X[1.4,c,m]}
		\tabucline[0.08em]-
		year           & ecologic value    & economic value  &cultural value         \\\tabucline[0.08em]-
		2010      &84.34\% & 10.41\% & 5.24\% \\
        2011      & 83.83\% & 10.72\% & 5.43\% \\
        2012      & 83.73\% & 10.68\% & 5.58\% \\
        2013      & 83.50\% & 10.66\% & 5.83\% \\
        2014	   & 83.76\% & 10.67\%   &6.03\% \\
		2015	   & 82.68\%   & 11.75\%   &5.57\% \\
		2016	   & 82.33\%    &12.55\%  &5.11\%   \\
        2017       & 81.87\%  & 12.68\%  &5.45\% \\
        2018      &  81.64\%  & 12.68\%  &5.67\% \\
		\tabucline[0.08em]-
	\end{tabu}
%   \vspace*{-5mm}              % 调整图像下间距
    \label{table6}
\end{table}

Based on the data in Tab. [\ref{table5}], we can calculate the proportion of different forest values in Heilongjiang from 2010 to 2018. We apply the decision tree algorithm to make the analysis about Tab. [\ref{table6}] below, and use the Target Oriented Forest Management Strategy to give proper suggestion on the future direction of forest development.All the conclusions and advice are stated below.



	\begin{itemize}
\item \textbf{Conclusion 1:} Heilongjiang forest has been in the intermediate stage since 2010 according to the decision tree algorithm. The economic value ratio was stable at around 10.5\% and the ecological value ratio was above 80\% for 4 years which indicates that Heilongjiang forest was protected well at that time without deforestation or overuse.
\item \textbf{Conclusion 2:} The transition point during 2010 to 2018 in Heilongjiang forest is 2015 when the economic value ratio was more than 10.8\% and the ecological value ratio was still above 80\%, a symbol of an advanced stage in decision tree algorithm. 
\item \textbf{Suggestion:} Heilongjiang forest can be seen as a stable and advanced forest since 2014 with few possibility that the proportion may change accidently, so the development strategy can be directly into culture-oriented one. Government should explore the underlying value of the forest in cultural layer such education, tourism and scientific research.
	\end{itemize}
    

%第二章:模型假设
%主要包括:模型的假设

\section{Case Study}

Since we are required to provide forest managers with reasonable forest management plans to understand the best use of forests, we combine \textbf{SHG Simulation Model} and\textbf{ EEC Value Model }with specific examples to comprehensively consider and calculate the ecological, economic and social values of forests. Then, we use the \textbf{Target Oriented Forest Management Strategy} to make sure harvest is included in the forest management plan and determine the future management plan most suitable for the current forest development.

\subsection{Forests of Heilongjiang Province,China}

Heilongjiang is located in Northeast China with vast forests which can bring huge benefits for local people. 

\begin{figure}[H]
\centering
\includegraphics[scale = 0.08]{mcmthesis-demo/figures/heilongjiang.png}
\caption{Forest distribution map of Heilongjiang Province} 
\end{figure}

We log on the official websites of China Meteorological Administration and Forestry Administration to obtain the data required in EEC Value Model from 2010 to 2018, and quantify the forest value of Heilongjiang according to its formula. As shown in the table below.

\begin{table}[h]
	\centering
%   \vspace*{-14cm}              % 调整图像上间距
	\caption{Quantitative forest value in Heilongjiang from 2014 to 2016(billion dollars)}
	\tabulinesep=1mm
	\begin{tabu}to \linewidth{X[0.9,c,m]X[1.4,c,m]X[1.4,c,m]X[1.4,c,m]}
		\tabucline[0.08em]-
		year           & ecologic value    & economic value  &cultural value         \\\tabucline[0.08em]-
		2010      & 55.9578 & 6.9057 & 3.47850 \\
        2011      & 57.2133 & 7.3199 & 3.7113 \\
        2012      & 59.7257 & 7.6231 & 3.9765 \\
        2013      & 61.7045 & 7.8787 & 4.3097 \\
        2014	   & 62.8302      & 8.0561    &4.54694  \\
		2015	   & 67.7652   & 9.6319    &4.5624  \\
		2016	   & 73.0542    &11.1379  &4.5402   \\
        2017       & 74.2948  & 11.5136  &4.9423 \\
        2018      &  75.3012  & 11.6933  &5.2304 \\
        
		\tabucline[0.08em]-
	\end{tabu}
%   \vspace*{-5mm}              % 调整图像下间距
    \label{table5}
\end{table}

\subsubsection{Carbon sequestration stock in the next 100 years}

According to the SHG Simulation Model, we get the maximum annual carbon dioxide storage under the optimal proportion of forest harvest. After accumulation, the total amount of carbon dioxide that can be stored in Heilongjiang forest within 100 years can be obtained. The annual $TCS$ and $\alpha$ is plotted in Fig. [\ref{Forecast line chart}].
\begin{figure}[H]
\centering
\includegraphics[height = 9cm, width = 14cm]{mcmthesis-demo/figures/Stock.png}
\caption{Forecast line chart of carbon sequestration and proportion of tree harvest}
\label{Forecast line chart}
\end{figure}
According to our model, when the initial $\alpha =0.01767 $ and $\beta = 2.286 $ ,the total amount of the carbon sequestration over 100 years will reach its maximum of $7.599\times 10^9$ tonnes, 15.6 \% more than carbon sequestration taken by untouched forest. So the harvest should be considered in the management plan to maximize the amount of annual carbon sequestration.

\subsubsection{Forest Management Plan in Heilongjiang Forest}

\begin{table}[b]
	\centering
%   \vspace*{-14cm}              % 调整图像上间距
	\caption{the proportion of different forest values in Heilongjiang from 2010 to 2018}
	\tabulinesep=1mm
	\begin{tabu}to \linewidth{X[0.9,c,m]X[1.4,c,m]X[1.4,c,m]X[1.4,c,m]}
		\tabucline[0.08em]-
		year           & ecologic value    & economic value  &cultural value         \\\tabucline[0.08em]-
		2010      &84.34\% & 10.41\% & 5.24\% \\
        2011      & 83.83\% & 10.72\% & 5.43\% \\
        2012      & 83.73\% & 10.68\% & 5.58\% \\
        2013      & 83.50\% & 10.66\% & 5.83\% \\
        2014	   & 83.76\% & 10.67\%   &6.03\% \\
		2015	   & 82.68\%   & 11.75\%   &5.57\% \\
		2016	   & 82.33\%    &12.55\%  &5.11\%   \\
        2017       & 81.87\%  & 12.68\%  &5.45\% \\
        2018      &  81.64\%  & 12.68\%  &5.67\% \\
		\tabucline[0.08em]-
	\end{tabu}
%   \vspace*{-5mm}              % 调整图像下间距
    \label{table6}
\end{table}

Based on the data in Tab. [\ref{table5}], we can calculate the proportion of different forest values in Heilongjiang from 2010 to 2018. We apply the decision tree algorithm to make the analysis about Tab. [\ref{table6}] below, and use the Target Oriented Forest Management Strategy to give proper suggestion on the future direction of forest development.All the conclusions and advice are stated below.



	\begin{itemize}
\item \textbf{Conclusion 1:} Heilongjiang forest has been in the intermediate stage since 2010 according to the decision tree algorithm. The economic value ratio was stable at around 10.5\% and the ecological value ratio was above 80\% for 4 years which indicates that Heilongjiang forest was protected well at that time without deforestation or overuse.
\item \textbf{Conclusion 2:} The transition point during 2010 to 2018 in Heilongjiang forest is 2015 when the economic value ratio was more than 10.8\% and the ecological value ratio was still above 80\%, a symbol of an advanced stage in decision tree algorithm. 
\item \textbf{Suggestion:} Heilongjiang forest can be seen as a stable and advanced forest since 2014 with few possibility that the proportion may change accidently, so the development strategy can be directly into culture-oriented one. Government should explore the underlying value of the forest in cultural layer such education, tourism and scientific research.
	\end{itemize}
    

%第三章:模型准备
%主要包括:记号说明、数据收集、数据清洗,以及其他与数据相关的点

\section{Case Study}

Since we are required to provide forest managers with reasonable forest management plans to understand the best use of forests, we combine \textbf{SHG Simulation Model} and\textbf{ EEC Value Model }with specific examples to comprehensively consider and calculate the ecological, economic and social values of forests. Then, we use the \textbf{Target Oriented Forest Management Strategy} to make sure harvest is included in the forest management plan and determine the future management plan most suitable for the current forest development.

\subsection{Forests of Heilongjiang Province,China}

Heilongjiang is located in Northeast China with vast forests which can bring huge benefits for local people. 

\begin{figure}[H]
\centering
\includegraphics[scale = 0.08]{mcmthesis-demo/figures/heilongjiang.png}
\caption{Forest distribution map of Heilongjiang Province} 
\end{figure}

We log on the official websites of China Meteorological Administration and Forestry Administration to obtain the data required in EEC Value Model from 2010 to 2018, and quantify the forest value of Heilongjiang according to its formula. As shown in the table below.

\begin{table}[h]
	\centering
%   \vspace*{-14cm}              % 调整图像上间距
	\caption{Quantitative forest value in Heilongjiang from 2014 to 2016(billion dollars)}
	\tabulinesep=1mm
	\begin{tabu}to \linewidth{X[0.9,c,m]X[1.4,c,m]X[1.4,c,m]X[1.4,c,m]}
		\tabucline[0.08em]-
		year           & ecologic value    & economic value  &cultural value         \\\tabucline[0.08em]-
		2010      & 55.9578 & 6.9057 & 3.47850 \\
        2011      & 57.2133 & 7.3199 & 3.7113 \\
        2012      & 59.7257 & 7.6231 & 3.9765 \\
        2013      & 61.7045 & 7.8787 & 4.3097 \\
        2014	   & 62.8302      & 8.0561    &4.54694  \\
		2015	   & 67.7652   & 9.6319    &4.5624  \\
		2016	   & 73.0542    &11.1379  &4.5402   \\
        2017       & 74.2948  & 11.5136  &4.9423 \\
        2018      &  75.3012  & 11.6933  &5.2304 \\
        
		\tabucline[0.08em]-
	\end{tabu}
%   \vspace*{-5mm}              % 调整图像下间距
    \label{table5}
\end{table}

\subsubsection{Carbon sequestration stock in the next 100 years}

According to the SHG Simulation Model, we get the maximum annual carbon dioxide storage under the optimal proportion of forest harvest. After accumulation, the total amount of carbon dioxide that can be stored in Heilongjiang forest within 100 years can be obtained. The annual $TCS$ and $\alpha$ is plotted in Fig. [\ref{Forecast line chart}].
\begin{figure}[H]
\centering
\includegraphics[height = 9cm, width = 14cm]{mcmthesis-demo/figures/Stock.png}
\caption{Forecast line chart of carbon sequestration and proportion of tree harvest}
\label{Forecast line chart}
\end{figure}
According to our model, when the initial $\alpha =0.01767 $ and $\beta = 2.286 $ ,the total amount of the carbon sequestration over 100 years will reach its maximum of $7.599\times 10^9$ tonnes, 15.6 \% more than carbon sequestration taken by untouched forest. So the harvest should be considered in the management plan to maximize the amount of annual carbon sequestration.

\subsubsection{Forest Management Plan in Heilongjiang Forest}

\begin{table}[b]
	\centering
%   \vspace*{-14cm}              % 调整图像上间距
	\caption{the proportion of different forest values in Heilongjiang from 2010 to 2018}
	\tabulinesep=1mm
	\begin{tabu}to \linewidth{X[0.9,c,m]X[1.4,c,m]X[1.4,c,m]X[1.4,c,m]}
		\tabucline[0.08em]-
		year           & ecologic value    & economic value  &cultural value         \\\tabucline[0.08em]-
		2010      &84.34\% & 10.41\% & 5.24\% \\
        2011      & 83.83\% & 10.72\% & 5.43\% \\
        2012      & 83.73\% & 10.68\% & 5.58\% \\
        2013      & 83.50\% & 10.66\% & 5.83\% \\
        2014	   & 83.76\% & 10.67\%   &6.03\% \\
		2015	   & 82.68\%   & 11.75\%   &5.57\% \\
		2016	   & 82.33\%    &12.55\%  &5.11\%   \\
        2017       & 81.87\%  & 12.68\%  &5.45\% \\
        2018      &  81.64\%  & 12.68\%  &5.67\% \\
		\tabucline[0.08em]-
	\end{tabu}
%   \vspace*{-5mm}              % 调整图像下间距
    \label{table6}
\end{table}

Based on the data in Tab. [\ref{table5}], we can calculate the proportion of different forest values in Heilongjiang from 2010 to 2018. We apply the decision tree algorithm to make the analysis about Tab. [\ref{table6}] below, and use the Target Oriented Forest Management Strategy to give proper suggestion on the future direction of forest development.All the conclusions and advice are stated below.



	\begin{itemize}
\item \textbf{Conclusion 1:} Heilongjiang forest has been in the intermediate stage since 2010 according to the decision tree algorithm. The economic value ratio was stable at around 10.5\% and the ecological value ratio was above 80\% for 4 years which indicates that Heilongjiang forest was protected well at that time without deforestation or overuse.
\item \textbf{Conclusion 2:} The transition point during 2010 to 2018 in Heilongjiang forest is 2015 when the economic value ratio was more than 10.8\% and the ecological value ratio was still above 80\%, a symbol of an advanced stage in decision tree algorithm. 
\item \textbf{Suggestion:} Heilongjiang forest can be seen as a stable and advanced forest since 2014 with few possibility that the proportion may change accidently, so the development strategy can be directly into culture-oriented one. Government should explore the underlying value of the forest in cultural layer such education, tourism and scientific research.
	\end{itemize}
    

%第四章:模型一建立与求解
%主要包括:模型一的一些实现过程、模型一的结果(对问题的解答)

\section{Case Study}

Since we are required to provide forest managers with reasonable forest management plans to understand the best use of forests, we combine \textbf{SHG Simulation Model} and\textbf{ EEC Value Model }with specific examples to comprehensively consider and calculate the ecological, economic and social values of forests. Then, we use the \textbf{Target Oriented Forest Management Strategy} to make sure harvest is included in the forest management plan and determine the future management plan most suitable for the current forest development.

\subsection{Forests of Heilongjiang Province,China}

Heilongjiang is located in Northeast China with vast forests which can bring huge benefits for local people. 

\begin{figure}[H]
\centering
\includegraphics[scale = 0.08]{mcmthesis-demo/figures/heilongjiang.png}
\caption{Forest distribution map of Heilongjiang Province} 
\end{figure}

We log on the official websites of China Meteorological Administration and Forestry Administration to obtain the data required in EEC Value Model from 2010 to 2018, and quantify the forest value of Heilongjiang according to its formula. As shown in the table below.

\begin{table}[h]
	\centering
%   \vspace*{-14cm}              % 调整图像上间距
	\caption{Quantitative forest value in Heilongjiang from 2014 to 2016(billion dollars)}
	\tabulinesep=1mm
	\begin{tabu}to \linewidth{X[0.9,c,m]X[1.4,c,m]X[1.4,c,m]X[1.4,c,m]}
		\tabucline[0.08em]-
		year           & ecologic value    & economic value  &cultural value         \\\tabucline[0.08em]-
		2010      & 55.9578 & 6.9057 & 3.47850 \\
        2011      & 57.2133 & 7.3199 & 3.7113 \\
        2012      & 59.7257 & 7.6231 & 3.9765 \\
        2013      & 61.7045 & 7.8787 & 4.3097 \\
        2014	   & 62.8302      & 8.0561    &4.54694  \\
		2015	   & 67.7652   & 9.6319    &4.5624  \\
		2016	   & 73.0542    &11.1379  &4.5402   \\
        2017       & 74.2948  & 11.5136  &4.9423 \\
        2018      &  75.3012  & 11.6933  &5.2304 \\
        
		\tabucline[0.08em]-
	\end{tabu}
%   \vspace*{-5mm}              % 调整图像下间距
    \label{table5}
\end{table}

\subsubsection{Carbon sequestration stock in the next 100 years}

According to the SHG Simulation Model, we get the maximum annual carbon dioxide storage under the optimal proportion of forest harvest. After accumulation, the total amount of carbon dioxide that can be stored in Heilongjiang forest within 100 years can be obtained. The annual $TCS$ and $\alpha$ is plotted in Fig. [\ref{Forecast line chart}].
\begin{figure}[H]
\centering
\includegraphics[height = 9cm, width = 14cm]{mcmthesis-demo/figures/Stock.png}
\caption{Forecast line chart of carbon sequestration and proportion of tree harvest}
\label{Forecast line chart}
\end{figure}
According to our model, when the initial $\alpha =0.01767 $ and $\beta = 2.286 $ ,the total amount of the carbon sequestration over 100 years will reach its maximum of $7.599\times 10^9$ tonnes, 15.6 \% more than carbon sequestration taken by untouched forest. So the harvest should be considered in the management plan to maximize the amount of annual carbon sequestration.

\subsubsection{Forest Management Plan in Heilongjiang Forest}

\begin{table}[b]
	\centering
%   \vspace*{-14cm}              % 调整图像上间距
	\caption{the proportion of different forest values in Heilongjiang from 2010 to 2018}
	\tabulinesep=1mm
	\begin{tabu}to \linewidth{X[0.9,c,m]X[1.4,c,m]X[1.4,c,m]X[1.4,c,m]}
		\tabucline[0.08em]-
		year           & ecologic value    & economic value  &cultural value         \\\tabucline[0.08em]-
		2010      &84.34\% & 10.41\% & 5.24\% \\
        2011      & 83.83\% & 10.72\% & 5.43\% \\
        2012      & 83.73\% & 10.68\% & 5.58\% \\
        2013      & 83.50\% & 10.66\% & 5.83\% \\
        2014	   & 83.76\% & 10.67\%   &6.03\% \\
		2015	   & 82.68\%   & 11.75\%   &5.57\% \\
		2016	   & 82.33\%    &12.55\%  &5.11\%   \\
        2017       & 81.87\%  & 12.68\%  &5.45\% \\
        2018      &  81.64\%  & 12.68\%  &5.67\% \\
		\tabucline[0.08em]-
	\end{tabu}
%   \vspace*{-5mm}              % 调整图像下间距
    \label{table6}
\end{table}

Based on the data in Tab. [\ref{table5}], we can calculate the proportion of different forest values in Heilongjiang from 2010 to 2018. We apply the decision tree algorithm to make the analysis about Tab. [\ref{table6}] below, and use the Target Oriented Forest Management Strategy to give proper suggestion on the future direction of forest development.All the conclusions and advice are stated below.



	\begin{itemize}
\item \textbf{Conclusion 1:} Heilongjiang forest has been in the intermediate stage since 2010 according to the decision tree algorithm. The economic value ratio was stable at around 10.5\% and the ecological value ratio was above 80\% for 4 years which indicates that Heilongjiang forest was protected well at that time without deforestation or overuse.
\item \textbf{Conclusion 2:} The transition point during 2010 to 2018 in Heilongjiang forest is 2015 when the economic value ratio was more than 10.8\% and the ecological value ratio was still above 80\%, a symbol of an advanced stage in decision tree algorithm. 
\item \textbf{Suggestion:} Heilongjiang forest can be seen as a stable and advanced forest since 2014 with few possibility that the proportion may change accidently, so the development strategy can be directly into culture-oriented one. Government should explore the underlying value of the forest in cultural layer such education, tourism and scientific research.
	\end{itemize}
    

%第五章:模型二建立与求解
%主要包括:模型儿的一些实现过程、模型二的结果(对问题的解答)

\section{Case Study}

Since we are required to provide forest managers with reasonable forest management plans to understand the best use of forests, we combine \textbf{SHG Simulation Model} and\textbf{ EEC Value Model }with specific examples to comprehensively consider and calculate the ecological, economic and social values of forests. Then, we use the \textbf{Target Oriented Forest Management Strategy} to make sure harvest is included in the forest management plan and determine the future management plan most suitable for the current forest development.

\subsection{Forests of Heilongjiang Province,China}

Heilongjiang is located in Northeast China with vast forests which can bring huge benefits for local people. 

\begin{figure}[H]
\centering
\includegraphics[scale = 0.08]{mcmthesis-demo/figures/heilongjiang.png}
\caption{Forest distribution map of Heilongjiang Province} 
\end{figure}

We log on the official websites of China Meteorological Administration and Forestry Administration to obtain the data required in EEC Value Model from 2010 to 2018, and quantify the forest value of Heilongjiang according to its formula. As shown in the table below.

\begin{table}[h]
	\centering
%   \vspace*{-14cm}              % 调整图像上间距
	\caption{Quantitative forest value in Heilongjiang from 2014 to 2016(billion dollars)}
	\tabulinesep=1mm
	\begin{tabu}to \linewidth{X[0.9,c,m]X[1.4,c,m]X[1.4,c,m]X[1.4,c,m]}
		\tabucline[0.08em]-
		year           & ecologic value    & economic value  &cultural value         \\\tabucline[0.08em]-
		2010      & 55.9578 & 6.9057 & 3.47850 \\
        2011      & 57.2133 & 7.3199 & 3.7113 \\
        2012      & 59.7257 & 7.6231 & 3.9765 \\
        2013      & 61.7045 & 7.8787 & 4.3097 \\
        2014	   & 62.8302      & 8.0561    &4.54694  \\
		2015	   & 67.7652   & 9.6319    &4.5624  \\
		2016	   & 73.0542    &11.1379  &4.5402   \\
        2017       & 74.2948  & 11.5136  &4.9423 \\
        2018      &  75.3012  & 11.6933  &5.2304 \\
        
		\tabucline[0.08em]-
	\end{tabu}
%   \vspace*{-5mm}              % 调整图像下间距
    \label{table5}
\end{table}

\subsubsection{Carbon sequestration stock in the next 100 years}

According to the SHG Simulation Model, we get the maximum annual carbon dioxide storage under the optimal proportion of forest harvest. After accumulation, the total amount of carbon dioxide that can be stored in Heilongjiang forest within 100 years can be obtained. The annual $TCS$ and $\alpha$ is plotted in Fig. [\ref{Forecast line chart}].
\begin{figure}[H]
\centering
\includegraphics[height = 9cm, width = 14cm]{mcmthesis-demo/figures/Stock.png}
\caption{Forecast line chart of carbon sequestration and proportion of tree harvest}
\label{Forecast line chart}
\end{figure}
According to our model, when the initial $\alpha =0.01767 $ and $\beta = 2.286 $ ,the total amount of the carbon sequestration over 100 years will reach its maximum of $7.599\times 10^9$ tonnes, 15.6 \% more than carbon sequestration taken by untouched forest. So the harvest should be considered in the management plan to maximize the amount of annual carbon sequestration.

\subsubsection{Forest Management Plan in Heilongjiang Forest}

\begin{table}[b]
	\centering
%   \vspace*{-14cm}              % 调整图像上间距
	\caption{the proportion of different forest values in Heilongjiang from 2010 to 2018}
	\tabulinesep=1mm
	\begin{tabu}to \linewidth{X[0.9,c,m]X[1.4,c,m]X[1.4,c,m]X[1.4,c,m]}
		\tabucline[0.08em]-
		year           & ecologic value    & economic value  &cultural value         \\\tabucline[0.08em]-
		2010      &84.34\% & 10.41\% & 5.24\% \\
        2011      & 83.83\% & 10.72\% & 5.43\% \\
        2012      & 83.73\% & 10.68\% & 5.58\% \\
        2013      & 83.50\% & 10.66\% & 5.83\% \\
        2014	   & 83.76\% & 10.67\%   &6.03\% \\
		2015	   & 82.68\%   & 11.75\%   &5.57\% \\
		2016	   & 82.33\%    &12.55\%  &5.11\%   \\
        2017       & 81.87\%  & 12.68\%  &5.45\% \\
        2018      &  81.64\%  & 12.68\%  &5.67\% \\
		\tabucline[0.08em]-
	\end{tabu}
%   \vspace*{-5mm}              % 调整图像下间距
    \label{table6}
\end{table}

Based on the data in Tab. [\ref{table5}], we can calculate the proportion of different forest values in Heilongjiang from 2010 to 2018. We apply the decision tree algorithm to make the analysis about Tab. [\ref{table6}] below, and use the Target Oriented Forest Management Strategy to give proper suggestion on the future direction of forest development.All the conclusions and advice are stated below.



	\begin{itemize}
\item \textbf{Conclusion 1:} Heilongjiang forest has been in the intermediate stage since 2010 according to the decision tree algorithm. The economic value ratio was stable at around 10.5\% and the ecological value ratio was above 80\% for 4 years which indicates that Heilongjiang forest was protected well at that time without deforestation or overuse.
\item \textbf{Conclusion 2:} The transition point during 2010 to 2018 in Heilongjiang forest is 2015 when the economic value ratio was more than 10.8\% and the ecological value ratio was still above 80\%, a symbol of an advanced stage in decision tree algorithm. 
\item \textbf{Suggestion:} Heilongjiang forest can be seen as a stable and advanced forest since 2014 with few possibility that the proportion may change accidently, so the development strategy can be directly into culture-oriented one. Government should explore the underlying value of the forest in cultural layer such education, tourism and scientific research.
	\end{itemize}
    

%第六章:模型三建立与求解
%主要包括:模型三的一些实现过程、模型三的结果(对问题的解答)

\section{Case Study}

Since we are required to provide forest managers with reasonable forest management plans to understand the best use of forests, we combine \textbf{SHG Simulation Model} and\textbf{ EEC Value Model }with specific examples to comprehensively consider and calculate the ecological, economic and social values of forests. Then, we use the \textbf{Target Oriented Forest Management Strategy} to make sure harvest is included in the forest management plan and determine the future management plan most suitable for the current forest development.

\subsection{Forests of Heilongjiang Province,China}

Heilongjiang is located in Northeast China with vast forests which can bring huge benefits for local people. 

\begin{figure}[H]
\centering
\includegraphics[scale = 0.08]{mcmthesis-demo/figures/heilongjiang.png}
\caption{Forest distribution map of Heilongjiang Province} 
\end{figure}

We log on the official websites of China Meteorological Administration and Forestry Administration to obtain the data required in EEC Value Model from 2010 to 2018, and quantify the forest value of Heilongjiang according to its formula. As shown in the table below.

\begin{table}[h]
	\centering
%   \vspace*{-14cm}              % 调整图像上间距
	\caption{Quantitative forest value in Heilongjiang from 2014 to 2016(billion dollars)}
	\tabulinesep=1mm
	\begin{tabu}to \linewidth{X[0.9,c,m]X[1.4,c,m]X[1.4,c,m]X[1.4,c,m]}
		\tabucline[0.08em]-
		year           & ecologic value    & economic value  &cultural value         \\\tabucline[0.08em]-
		2010      & 55.9578 & 6.9057 & 3.47850 \\
        2011      & 57.2133 & 7.3199 & 3.7113 \\
        2012      & 59.7257 & 7.6231 & 3.9765 \\
        2013      & 61.7045 & 7.8787 & 4.3097 \\
        2014	   & 62.8302      & 8.0561    &4.54694  \\
		2015	   & 67.7652   & 9.6319    &4.5624  \\
		2016	   & 73.0542    &11.1379  &4.5402   \\
        2017       & 74.2948  & 11.5136  &4.9423 \\
        2018      &  75.3012  & 11.6933  &5.2304 \\
        
		\tabucline[0.08em]-
	\end{tabu}
%   \vspace*{-5mm}              % 调整图像下间距
    \label{table5}
\end{table}

\subsubsection{Carbon sequestration stock in the next 100 years}

According to the SHG Simulation Model, we get the maximum annual carbon dioxide storage under the optimal proportion of forest harvest. After accumulation, the total amount of carbon dioxide that can be stored in Heilongjiang forest within 100 years can be obtained. The annual $TCS$ and $\alpha$ is plotted in Fig. [\ref{Forecast line chart}].
\begin{figure}[H]
\centering
\includegraphics[height = 9cm, width = 14cm]{mcmthesis-demo/figures/Stock.png}
\caption{Forecast line chart of carbon sequestration and proportion of tree harvest}
\label{Forecast line chart}
\end{figure}
According to our model, when the initial $\alpha =0.01767 $ and $\beta = 2.286 $ ,the total amount of the carbon sequestration over 100 years will reach its maximum of $7.599\times 10^9$ tonnes, 15.6 \% more than carbon sequestration taken by untouched forest. So the harvest should be considered in the management plan to maximize the amount of annual carbon sequestration.

\subsubsection{Forest Management Plan in Heilongjiang Forest}

\begin{table}[b]
	\centering
%   \vspace*{-14cm}              % 调整图像上间距
	\caption{the proportion of different forest values in Heilongjiang from 2010 to 2018}
	\tabulinesep=1mm
	\begin{tabu}to \linewidth{X[0.9,c,m]X[1.4,c,m]X[1.4,c,m]X[1.4,c,m]}
		\tabucline[0.08em]-
		year           & ecologic value    & economic value  &cultural value         \\\tabucline[0.08em]-
		2010      &84.34\% & 10.41\% & 5.24\% \\
        2011      & 83.83\% & 10.72\% & 5.43\% \\
        2012      & 83.73\% & 10.68\% & 5.58\% \\
        2013      & 83.50\% & 10.66\% & 5.83\% \\
        2014	   & 83.76\% & 10.67\%   &6.03\% \\
		2015	   & 82.68\%   & 11.75\%   &5.57\% \\
		2016	   & 82.33\%    &12.55\%  &5.11\%   \\
        2017       & 81.87\%  & 12.68\%  &5.45\% \\
        2018      &  81.64\%  & 12.68\%  &5.67\% \\
		\tabucline[0.08em]-
	\end{tabu}
%   \vspace*{-5mm}              % 调整图像下间距
    \label{table6}
\end{table}

Based on the data in Tab. [\ref{table5}], we can calculate the proportion of different forest values in Heilongjiang from 2010 to 2018. We apply the decision tree algorithm to make the analysis about Tab. [\ref{table6}] below, and use the Target Oriented Forest Management Strategy to give proper suggestion on the future direction of forest development.All the conclusions and advice are stated below.



	\begin{itemize}
\item \textbf{Conclusion 1:} Heilongjiang forest has been in the intermediate stage since 2010 according to the decision tree algorithm. The economic value ratio was stable at around 10.5\% and the ecological value ratio was above 80\% for 4 years which indicates that Heilongjiang forest was protected well at that time without deforestation or overuse.
\item \textbf{Conclusion 2:} The transition point during 2010 to 2018 in Heilongjiang forest is 2015 when the economic value ratio was more than 10.8\% and the ecological value ratio was still above 80\%, a symbol of an advanced stage in decision tree algorithm. 
\item \textbf{Suggestion:} Heilongjiang forest can be seen as a stable and advanced forest since 2014 with few possibility that the proportion may change accidently, so the development strategy can be directly into culture-oriented one. Government should explore the underlying value of the forest in cultural layer such education, tourism and scientific research.
	\end{itemize}
    

%第七章:模型检验
%主要包括:灵敏性检验、鲁棒性检验

\section{Case Study}

Since we are required to provide forest managers with reasonable forest management plans to understand the best use of forests, we combine \textbf{SHG Simulation Model} and\textbf{ EEC Value Model }with specific examples to comprehensively consider and calculate the ecological, economic and social values of forests. Then, we use the \textbf{Target Oriented Forest Management Strategy} to make sure harvest is included in the forest management plan and determine the future management plan most suitable for the current forest development.

\subsection{Forests of Heilongjiang Province,China}

Heilongjiang is located in Northeast China with vast forests which can bring huge benefits for local people. 

\begin{figure}[H]
\centering
\includegraphics[scale = 0.08]{mcmthesis-demo/figures/heilongjiang.png}
\caption{Forest distribution map of Heilongjiang Province} 
\end{figure}

We log on the official websites of China Meteorological Administration and Forestry Administration to obtain the data required in EEC Value Model from 2010 to 2018, and quantify the forest value of Heilongjiang according to its formula. As shown in the table below.

\begin{table}[h]
	\centering
%   \vspace*{-14cm}              % 调整图像上间距
	\caption{Quantitative forest value in Heilongjiang from 2014 to 2016(billion dollars)}
	\tabulinesep=1mm
	\begin{tabu}to \linewidth{X[0.9,c,m]X[1.4,c,m]X[1.4,c,m]X[1.4,c,m]}
		\tabucline[0.08em]-
		year           & ecologic value    & economic value  &cultural value         \\\tabucline[0.08em]-
		2010      & 55.9578 & 6.9057 & 3.47850 \\
        2011      & 57.2133 & 7.3199 & 3.7113 \\
        2012      & 59.7257 & 7.6231 & 3.9765 \\
        2013      & 61.7045 & 7.8787 & 4.3097 \\
        2014	   & 62.8302      & 8.0561    &4.54694  \\
		2015	   & 67.7652   & 9.6319    &4.5624  \\
		2016	   & 73.0542    &11.1379  &4.5402   \\
        2017       & 74.2948  & 11.5136  &4.9423 \\
        2018      &  75.3012  & 11.6933  &5.2304 \\
        
		\tabucline[0.08em]-
	\end{tabu}
%   \vspace*{-5mm}              % 调整图像下间距
    \label{table5}
\end{table}

\subsubsection{Carbon sequestration stock in the next 100 years}

According to the SHG Simulation Model, we get the maximum annual carbon dioxide storage under the optimal proportion of forest harvest. After accumulation, the total amount of carbon dioxide that can be stored in Heilongjiang forest within 100 years can be obtained. The annual $TCS$ and $\alpha$ is plotted in Fig. [\ref{Forecast line chart}].
\begin{figure}[H]
\centering
\includegraphics[height = 9cm, width = 14cm]{mcmthesis-demo/figures/Stock.png}
\caption{Forecast line chart of carbon sequestration and proportion of tree harvest}
\label{Forecast line chart}
\end{figure}
According to our model, when the initial $\alpha =0.01767 $ and $\beta = 2.286 $ ,the total amount of the carbon sequestration over 100 years will reach its maximum of $7.599\times 10^9$ tonnes, 15.6 \% more than carbon sequestration taken by untouched forest. So the harvest should be considered in the management plan to maximize the amount of annual carbon sequestration.

\subsubsection{Forest Management Plan in Heilongjiang Forest}

\begin{table}[b]
	\centering
%   \vspace*{-14cm}              % 调整图像上间距
	\caption{the proportion of different forest values in Heilongjiang from 2010 to 2018}
	\tabulinesep=1mm
	\begin{tabu}to \linewidth{X[0.9,c,m]X[1.4,c,m]X[1.4,c,m]X[1.4,c,m]}
		\tabucline[0.08em]-
		year           & ecologic value    & economic value  &cultural value         \\\tabucline[0.08em]-
		2010      &84.34\% & 10.41\% & 5.24\% \\
        2011      & 83.83\% & 10.72\% & 5.43\% \\
        2012      & 83.73\% & 10.68\% & 5.58\% \\
        2013      & 83.50\% & 10.66\% & 5.83\% \\
        2014	   & 83.76\% & 10.67\%   &6.03\% \\
		2015	   & 82.68\%   & 11.75\%   &5.57\% \\
		2016	   & 82.33\%    &12.55\%  &5.11\%   \\
        2017       & 81.87\%  & 12.68\%  &5.45\% \\
        2018      &  81.64\%  & 12.68\%  &5.67\% \\
		\tabucline[0.08em]-
	\end{tabu}
%   \vspace*{-5mm}              % 调整图像下间距
    \label{table6}
\end{table}

Based on the data in Tab. [\ref{table5}], we can calculate the proportion of different forest values in Heilongjiang from 2010 to 2018. We apply the decision tree algorithm to make the analysis about Tab. [\ref{table6}] below, and use the Target Oriented Forest Management Strategy to give proper suggestion on the future direction of forest development.All the conclusions and advice are stated below.



	\begin{itemize}
\item \textbf{Conclusion 1:} Heilongjiang forest has been in the intermediate stage since 2010 according to the decision tree algorithm. The economic value ratio was stable at around 10.5\% and the ecological value ratio was above 80\% for 4 years which indicates that Heilongjiang forest was protected well at that time without deforestation or overuse.
\item \textbf{Conclusion 2:} The transition point during 2010 to 2018 in Heilongjiang forest is 2015 when the economic value ratio was more than 10.8\% and the ecological value ratio was still above 80\%, a symbol of an advanced stage in decision tree algorithm. 
\item \textbf{Suggestion:} Heilongjiang forest can be seen as a stable and advanced forest since 2014 with few possibility that the proportion may change accidently, so the development strategy can be directly into culture-oriented one. Government should explore the underlying value of the forest in cultural layer such education, tourism and scientific research.
	\end{itemize}
    

%第八章:结论
%主要包括:问题的解决方案
% \newpage
\section{Case Study}

Since we are required to provide forest managers with reasonable forest management plans to understand the best use of forests, we combine \textbf{SHG Simulation Model} and\textbf{ EEC Value Model }with specific examples to comprehensively consider and calculate the ecological, economic and social values of forests. Then, we use the \textbf{Target Oriented Forest Management Strategy} to make sure harvest is included in the forest management plan and determine the future management plan most suitable for the current forest development.

\subsection{Forests of Heilongjiang Province,China}

Heilongjiang is located in Northeast China with vast forests which can bring huge benefits for local people. 

\begin{figure}[H]
\centering
\includegraphics[scale = 0.08]{mcmthesis-demo/figures/heilongjiang.png}
\caption{Forest distribution map of Heilongjiang Province} 
\end{figure}

We log on the official websites of China Meteorological Administration and Forestry Administration to obtain the data required in EEC Value Model from 2010 to 2018, and quantify the forest value of Heilongjiang according to its formula. As shown in the table below.

\begin{table}[h]
	\centering
%   \vspace*{-14cm}              % 调整图像上间距
	\caption{Quantitative forest value in Heilongjiang from 2014 to 2016(billion dollars)}
	\tabulinesep=1mm
	\begin{tabu}to \linewidth{X[0.9,c,m]X[1.4,c,m]X[1.4,c,m]X[1.4,c,m]}
		\tabucline[0.08em]-
		year           & ecologic value    & economic value  &cultural value         \\\tabucline[0.08em]-
		2010      & 55.9578 & 6.9057 & 3.47850 \\
        2011      & 57.2133 & 7.3199 & 3.7113 \\
        2012      & 59.7257 & 7.6231 & 3.9765 \\
        2013      & 61.7045 & 7.8787 & 4.3097 \\
        2014	   & 62.8302      & 8.0561    &4.54694  \\
		2015	   & 67.7652   & 9.6319    &4.5624  \\
		2016	   & 73.0542    &11.1379  &4.5402   \\
        2017       & 74.2948  & 11.5136  &4.9423 \\
        2018      &  75.3012  & 11.6933  &5.2304 \\
        
		\tabucline[0.08em]-
	\end{tabu}
%   \vspace*{-5mm}              % 调整图像下间距
    \label{table5}
\end{table}

\subsubsection{Carbon sequestration stock in the next 100 years}

According to the SHG Simulation Model, we get the maximum annual carbon dioxide storage under the optimal proportion of forest harvest. After accumulation, the total amount of carbon dioxide that can be stored in Heilongjiang forest within 100 years can be obtained. The annual $TCS$ and $\alpha$ is plotted in Fig. [\ref{Forecast line chart}].
\begin{figure}[H]
\centering
\includegraphics[height = 9cm, width = 14cm]{mcmthesis-demo/figures/Stock.png}
\caption{Forecast line chart of carbon sequestration and proportion of tree harvest}
\label{Forecast line chart}
\end{figure}
According to our model, when the initial $\alpha =0.01767 $ and $\beta = 2.286 $ ,the total amount of the carbon sequestration over 100 years will reach its maximum of $7.599\times 10^9$ tonnes, 15.6 \% more than carbon sequestration taken by untouched forest. So the harvest should be considered in the management plan to maximize the amount of annual carbon sequestration.

\subsubsection{Forest Management Plan in Heilongjiang Forest}

\begin{table}[b]
	\centering
%   \vspace*{-14cm}              % 调整图像上间距
	\caption{the proportion of different forest values in Heilongjiang from 2010 to 2018}
	\tabulinesep=1mm
	\begin{tabu}to \linewidth{X[0.9,c,m]X[1.4,c,m]X[1.4,c,m]X[1.4,c,m]}
		\tabucline[0.08em]-
		year           & ecologic value    & economic value  &cultural value         \\\tabucline[0.08em]-
		2010      &84.34\% & 10.41\% & 5.24\% \\
        2011      & 83.83\% & 10.72\% & 5.43\% \\
        2012      & 83.73\% & 10.68\% & 5.58\% \\
        2013      & 83.50\% & 10.66\% & 5.83\% \\
        2014	   & 83.76\% & 10.67\%   &6.03\% \\
		2015	   & 82.68\%   & 11.75\%   &5.57\% \\
		2016	   & 82.33\%    &12.55\%  &5.11\%   \\
        2017       & 81.87\%  & 12.68\%  &5.45\% \\
        2018      &  81.64\%  & 12.68\%  &5.67\% \\
		\tabucline[0.08em]-
	\end{tabu}
%   \vspace*{-5mm}              % 调整图像下间距
    \label{table6}
\end{table}

Based on the data in Tab. [\ref{table5}], we can calculate the proportion of different forest values in Heilongjiang from 2010 to 2018. We apply the decision tree algorithm to make the analysis about Tab. [\ref{table6}] below, and use the Target Oriented Forest Management Strategy to give proper suggestion on the future direction of forest development.All the conclusions and advice are stated below.



	\begin{itemize}
\item \textbf{Conclusion 1:} Heilongjiang forest has been in the intermediate stage since 2010 according to the decision tree algorithm. The economic value ratio was stable at around 10.5\% and the ecological value ratio was above 80\% for 4 years which indicates that Heilongjiang forest was protected well at that time without deforestation or overuse.
\item \textbf{Conclusion 2:} The transition point during 2010 to 2018 in Heilongjiang forest is 2015 when the economic value ratio was more than 10.8\% and the ecological value ratio was still above 80\%, a symbol of an advanced stage in decision tree algorithm. 
\item \textbf{Suggestion:} Heilongjiang forest can be seen as a stable and advanced forest since 2014 with few possibility that the proportion may change accidently, so the development strategy can be directly into culture-oriented one. Government should explore the underlying value of the forest in cultural layer such education, tourism and scientific research.
	\end{itemize}
    

%参考文献
\begin{appendices}
\vspace{-0.5cm}
\section{First appendix: Article for newspaper}

\newpage
\begin{figure}[H]
\centering
\setlength{\abovecaptionskip}{0.cm}
\includegraphics[width = \textwidth]{mcmthesis-demo/figures/Nontechnical Article.pdf}
\end{figure}
% \newpage

% \section{Second appendix: Ratio of wood products in different continents}

% \begin{table}[h]
% 	\centering
% %   \vspace*{-14cm}              % 调整图像上间距
% 	\caption{Ratio of wood products in different continents \cite{article}}
% 	\tabulinesep=1mm
% 	\begin{tabu}to \linewidth{X[c,m]X[2.5,c,m]X[c,m]X[c,m]X[c,m]X[c,m]X[c,m]X[c,m]}
% 		\tabucline[0.08em]-
% 		$i$ & $Item$                                  & $pp_{i, AF}$&  $pp_{i, AS}$ &  $pp_{i, EU}$ & $pp_{i, NA}$ & $pp_{i, Oce}$ & $pp_{i, SA}$\\ \tabucline[0.08em]-
% 		1	& Industrial roundwood                    & 10.95\%       &   0.1909\%            &     0.4947\%          &       0.5383\%       &     0.5998\%          &0.3604\%\\
% 		2	& Wood pellets and other agglomerates     & 0.0002\%	        &  0.0031\%             &    0.0262\%           &    0.0131\%          &  0.0029\%             &0.0113\%\\
% 		3	& Sawnwood                                & 0.0115\%	    &    0.0523\%           &     0.1256\%          &      0.0985\%        &   0.0898\%            &0.0588\%\\
% 		4	& Wood-based panel                        & 0.0021\%	        &   0.0393\%            &     0.0395\%          &    0.0372\%          &   0.0257\%            &0.0182\%\\
% 		5	& Papers                                  & 0.0141\%        &   0.1385\%            &     0.1836\%          &   0.2522\%           &      0.1373\%         &0.0918\%\\
% 		\tabucline[0.08em]-
% 	\end{tabu}
% %   \vspace*{-5mm}              % 调整图像下间距
%     \label{table}
% \end{table}

\end{appendices}
\end{document}
%% 
%% This work consists of these files mcmthesis.dtx,
%%                                   figures/ and
%%                                   code/,
%% and the derived files             mcmthesis.cls,
%%                                   mcmthesis-demo.tex,
%%                                   README,
%%                                   LICENSE,
%%                                   mcmthesis.pdf and
%%                                   mcmthesis-demo.pdf.
%%
%% End of file `mcmthesis-demo.tex'.

